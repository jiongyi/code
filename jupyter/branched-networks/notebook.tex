
% Default to the notebook output style

    


% Inherit from the specified cell style.




    
\documentclass[11pt]{article}

    
    
    \usepackage[T1]{fontenc}
    % Nicer default font (+ math font) than Computer Modern for most use cases
    \usepackage{mathpazo}

    % Basic figure setup, for now with no caption control since it's done
    % automatically by Pandoc (which extracts ![](path) syntax from Markdown).
    \usepackage{graphicx}
    % We will generate all images so they have a width \maxwidth. This means
    % that they will get their normal width if they fit onto the page, but
    % are scaled down if they would overflow the margins.
    \makeatletter
    \def\maxwidth{\ifdim\Gin@nat@width>\linewidth\linewidth
    \else\Gin@nat@width\fi}
    \makeatother
    \let\Oldincludegraphics\includegraphics
    % Set max figure width to be 80% of text width, for now hardcoded.
    \renewcommand{\includegraphics}[1]{\Oldincludegraphics[width=.8\maxwidth]{#1}}
    % Ensure that by default, figures have no caption (until we provide a
    % proper Figure object with a Caption API and a way to capture that
    % in the conversion process - todo).
    \usepackage{caption}
    \DeclareCaptionLabelFormat{nolabel}{}
    \captionsetup{labelformat=nolabel}

    \usepackage{adjustbox} % Used to constrain images to a maximum size 
    \usepackage{xcolor} % Allow colors to be defined
    \usepackage{enumerate} % Needed for markdown enumerations to work
    \usepackage{geometry} % Used to adjust the document margins
    \usepackage{amsmath} % Equations
    \usepackage{amssymb} % Equations
    \usepackage{textcomp} % defines textquotesingle
    % Hack from http://tex.stackexchange.com/a/47451/13684:
    \AtBeginDocument{%
        \def\PYZsq{\textquotesingle}% Upright quotes in Pygmentized code
    }
    \usepackage{upquote} % Upright quotes for verbatim code
    \usepackage{eurosym} % defines \euro
    \usepackage[mathletters]{ucs} % Extended unicode (utf-8) support
    \usepackage[utf8x]{inputenc} % Allow utf-8 characters in the tex document
    \usepackage{fancyvrb} % verbatim replacement that allows latex
    \usepackage{grffile} % extends the file name processing of package graphics 
                         % to support a larger range 
    % The hyperref package gives us a pdf with properly built
    % internal navigation ('pdf bookmarks' for the table of contents,
    % internal cross-reference links, web links for URLs, etc.)
    \usepackage{hyperref}
    \usepackage{longtable} % longtable support required by pandoc >1.10
    \usepackage{booktabs}  % table support for pandoc > 1.12.2
    \usepackage[inline]{enumitem} % IRkernel/repr support (it uses the enumerate* environment)
    \usepackage[normalem]{ulem} % ulem is needed to support strikethroughs (\sout)
                                % normalem makes italics be italics, not underlines
    

    
    
    % Colors for the hyperref package
    \definecolor{urlcolor}{rgb}{0,.145,.698}
    \definecolor{linkcolor}{rgb}{.71,0.21,0.01}
    \definecolor{citecolor}{rgb}{.12,.54,.11}

    % ANSI colors
    \definecolor{ansi-black}{HTML}{3E424D}
    \definecolor{ansi-black-intense}{HTML}{282C36}
    \definecolor{ansi-red}{HTML}{E75C58}
    \definecolor{ansi-red-intense}{HTML}{B22B31}
    \definecolor{ansi-green}{HTML}{00A250}
    \definecolor{ansi-green-intense}{HTML}{007427}
    \definecolor{ansi-yellow}{HTML}{DDB62B}
    \definecolor{ansi-yellow-intense}{HTML}{B27D12}
    \definecolor{ansi-blue}{HTML}{208FFB}
    \definecolor{ansi-blue-intense}{HTML}{0065CA}
    \definecolor{ansi-magenta}{HTML}{D160C4}
    \definecolor{ansi-magenta-intense}{HTML}{A03196}
    \definecolor{ansi-cyan}{HTML}{60C6C8}
    \definecolor{ansi-cyan-intense}{HTML}{258F8F}
    \definecolor{ansi-white}{HTML}{C5C1B4}
    \definecolor{ansi-white-intense}{HTML}{A1A6B2}

    % commands and environments needed by pandoc snippets
    % extracted from the output of `pandoc -s`
    \providecommand{\tightlist}{%
      \setlength{\itemsep}{0pt}\setlength{\parskip}{0pt}}
    \DefineVerbatimEnvironment{Highlighting}{Verbatim}{commandchars=\\\{\}}
    % Add ',fontsize=\small' for more characters per line
    \newenvironment{Shaded}{}{}
    \newcommand{\KeywordTok}[1]{\textcolor[rgb]{0.00,0.44,0.13}{\textbf{{#1}}}}
    \newcommand{\DataTypeTok}[1]{\textcolor[rgb]{0.56,0.13,0.00}{{#1}}}
    \newcommand{\DecValTok}[1]{\textcolor[rgb]{0.25,0.63,0.44}{{#1}}}
    \newcommand{\BaseNTok}[1]{\textcolor[rgb]{0.25,0.63,0.44}{{#1}}}
    \newcommand{\FloatTok}[1]{\textcolor[rgb]{0.25,0.63,0.44}{{#1}}}
    \newcommand{\CharTok}[1]{\textcolor[rgb]{0.25,0.44,0.63}{{#1}}}
    \newcommand{\StringTok}[1]{\textcolor[rgb]{0.25,0.44,0.63}{{#1}}}
    \newcommand{\CommentTok}[1]{\textcolor[rgb]{0.38,0.63,0.69}{\textit{{#1}}}}
    \newcommand{\OtherTok}[1]{\textcolor[rgb]{0.00,0.44,0.13}{{#1}}}
    \newcommand{\AlertTok}[1]{\textcolor[rgb]{1.00,0.00,0.00}{\textbf{{#1}}}}
    \newcommand{\FunctionTok}[1]{\textcolor[rgb]{0.02,0.16,0.49}{{#1}}}
    \newcommand{\RegionMarkerTok}[1]{{#1}}
    \newcommand{\ErrorTok}[1]{\textcolor[rgb]{1.00,0.00,0.00}{\textbf{{#1}}}}
    \newcommand{\NormalTok}[1]{{#1}}
    
    % Additional commands for more recent versions of Pandoc
    \newcommand{\ConstantTok}[1]{\textcolor[rgb]{0.53,0.00,0.00}{{#1}}}
    \newcommand{\SpecialCharTok}[1]{\textcolor[rgb]{0.25,0.44,0.63}{{#1}}}
    \newcommand{\VerbatimStringTok}[1]{\textcolor[rgb]{0.25,0.44,0.63}{{#1}}}
    \newcommand{\SpecialStringTok}[1]{\textcolor[rgb]{0.73,0.40,0.53}{{#1}}}
    \newcommand{\ImportTok}[1]{{#1}}
    \newcommand{\DocumentationTok}[1]{\textcolor[rgb]{0.73,0.13,0.13}{\textit{{#1}}}}
    \newcommand{\AnnotationTok}[1]{\textcolor[rgb]{0.38,0.63,0.69}{\textbf{\textit{{#1}}}}}
    \newcommand{\CommentVarTok}[1]{\textcolor[rgb]{0.38,0.63,0.69}{\textbf{\textit{{#1}}}}}
    \newcommand{\VariableTok}[1]{\textcolor[rgb]{0.10,0.09,0.49}{{#1}}}
    \newcommand{\ControlFlowTok}[1]{\textcolor[rgb]{0.00,0.44,0.13}{\textbf{{#1}}}}
    \newcommand{\OperatorTok}[1]{\textcolor[rgb]{0.40,0.40,0.40}{{#1}}}
    \newcommand{\BuiltInTok}[1]{{#1}}
    \newcommand{\ExtensionTok}[1]{{#1}}
    \newcommand{\PreprocessorTok}[1]{\textcolor[rgb]{0.74,0.48,0.00}{{#1}}}
    \newcommand{\AttributeTok}[1]{\textcolor[rgb]{0.49,0.56,0.16}{{#1}}}
    \newcommand{\InformationTok}[1]{\textcolor[rgb]{0.38,0.63,0.69}{\textbf{\textit{{#1}}}}}
    \newcommand{\WarningTok}[1]{\textcolor[rgb]{0.38,0.63,0.69}{\textbf{\textit{{#1}}}}}
    
    
    % Define a nice break command that doesn't care if a line doesn't already
    % exist.
    \def\br{\hspace*{\fill} \\* }
    % Math Jax compatability definitions
    \def\gt{>}
    \def\lt{<}
    % Document parameters
    \title{network-density-mechanics}
    
    
    

    % Pygments definitions
    
\makeatletter
\def\PY@reset{\let\PY@it=\relax \let\PY@bf=\relax%
    \let\PY@ul=\relax \let\PY@tc=\relax%
    \let\PY@bc=\relax \let\PY@ff=\relax}
\def\PY@tok#1{\csname PY@tok@#1\endcsname}
\def\PY@toks#1+{\ifx\relax#1\empty\else%
    \PY@tok{#1}\expandafter\PY@toks\fi}
\def\PY@do#1{\PY@bc{\PY@tc{\PY@ul{%
    \PY@it{\PY@bf{\PY@ff{#1}}}}}}}
\def\PY#1#2{\PY@reset\PY@toks#1+\relax+\PY@do{#2}}

\expandafter\def\csname PY@tok@w\endcsname{\def\PY@tc##1{\textcolor[rgb]{0.73,0.73,0.73}{##1}}}
\expandafter\def\csname PY@tok@c\endcsname{\let\PY@it=\textit\def\PY@tc##1{\textcolor[rgb]{0.25,0.50,0.50}{##1}}}
\expandafter\def\csname PY@tok@cp\endcsname{\def\PY@tc##1{\textcolor[rgb]{0.74,0.48,0.00}{##1}}}
\expandafter\def\csname PY@tok@k\endcsname{\let\PY@bf=\textbf\def\PY@tc##1{\textcolor[rgb]{0.00,0.50,0.00}{##1}}}
\expandafter\def\csname PY@tok@kp\endcsname{\def\PY@tc##1{\textcolor[rgb]{0.00,0.50,0.00}{##1}}}
\expandafter\def\csname PY@tok@kt\endcsname{\def\PY@tc##1{\textcolor[rgb]{0.69,0.00,0.25}{##1}}}
\expandafter\def\csname PY@tok@o\endcsname{\def\PY@tc##1{\textcolor[rgb]{0.40,0.40,0.40}{##1}}}
\expandafter\def\csname PY@tok@ow\endcsname{\let\PY@bf=\textbf\def\PY@tc##1{\textcolor[rgb]{0.67,0.13,1.00}{##1}}}
\expandafter\def\csname PY@tok@nb\endcsname{\def\PY@tc##1{\textcolor[rgb]{0.00,0.50,0.00}{##1}}}
\expandafter\def\csname PY@tok@nf\endcsname{\def\PY@tc##1{\textcolor[rgb]{0.00,0.00,1.00}{##1}}}
\expandafter\def\csname PY@tok@nc\endcsname{\let\PY@bf=\textbf\def\PY@tc##1{\textcolor[rgb]{0.00,0.00,1.00}{##1}}}
\expandafter\def\csname PY@tok@nn\endcsname{\let\PY@bf=\textbf\def\PY@tc##1{\textcolor[rgb]{0.00,0.00,1.00}{##1}}}
\expandafter\def\csname PY@tok@ne\endcsname{\let\PY@bf=\textbf\def\PY@tc##1{\textcolor[rgb]{0.82,0.25,0.23}{##1}}}
\expandafter\def\csname PY@tok@nv\endcsname{\def\PY@tc##1{\textcolor[rgb]{0.10,0.09,0.49}{##1}}}
\expandafter\def\csname PY@tok@no\endcsname{\def\PY@tc##1{\textcolor[rgb]{0.53,0.00,0.00}{##1}}}
\expandafter\def\csname PY@tok@nl\endcsname{\def\PY@tc##1{\textcolor[rgb]{0.63,0.63,0.00}{##1}}}
\expandafter\def\csname PY@tok@ni\endcsname{\let\PY@bf=\textbf\def\PY@tc##1{\textcolor[rgb]{0.60,0.60,0.60}{##1}}}
\expandafter\def\csname PY@tok@na\endcsname{\def\PY@tc##1{\textcolor[rgb]{0.49,0.56,0.16}{##1}}}
\expandafter\def\csname PY@tok@nt\endcsname{\let\PY@bf=\textbf\def\PY@tc##1{\textcolor[rgb]{0.00,0.50,0.00}{##1}}}
\expandafter\def\csname PY@tok@nd\endcsname{\def\PY@tc##1{\textcolor[rgb]{0.67,0.13,1.00}{##1}}}
\expandafter\def\csname PY@tok@s\endcsname{\def\PY@tc##1{\textcolor[rgb]{0.73,0.13,0.13}{##1}}}
\expandafter\def\csname PY@tok@sd\endcsname{\let\PY@it=\textit\def\PY@tc##1{\textcolor[rgb]{0.73,0.13,0.13}{##1}}}
\expandafter\def\csname PY@tok@si\endcsname{\let\PY@bf=\textbf\def\PY@tc##1{\textcolor[rgb]{0.73,0.40,0.53}{##1}}}
\expandafter\def\csname PY@tok@se\endcsname{\let\PY@bf=\textbf\def\PY@tc##1{\textcolor[rgb]{0.73,0.40,0.13}{##1}}}
\expandafter\def\csname PY@tok@sr\endcsname{\def\PY@tc##1{\textcolor[rgb]{0.73,0.40,0.53}{##1}}}
\expandafter\def\csname PY@tok@ss\endcsname{\def\PY@tc##1{\textcolor[rgb]{0.10,0.09,0.49}{##1}}}
\expandafter\def\csname PY@tok@sx\endcsname{\def\PY@tc##1{\textcolor[rgb]{0.00,0.50,0.00}{##1}}}
\expandafter\def\csname PY@tok@m\endcsname{\def\PY@tc##1{\textcolor[rgb]{0.40,0.40,0.40}{##1}}}
\expandafter\def\csname PY@tok@gh\endcsname{\let\PY@bf=\textbf\def\PY@tc##1{\textcolor[rgb]{0.00,0.00,0.50}{##1}}}
\expandafter\def\csname PY@tok@gu\endcsname{\let\PY@bf=\textbf\def\PY@tc##1{\textcolor[rgb]{0.50,0.00,0.50}{##1}}}
\expandafter\def\csname PY@tok@gd\endcsname{\def\PY@tc##1{\textcolor[rgb]{0.63,0.00,0.00}{##1}}}
\expandafter\def\csname PY@tok@gi\endcsname{\def\PY@tc##1{\textcolor[rgb]{0.00,0.63,0.00}{##1}}}
\expandafter\def\csname PY@tok@gr\endcsname{\def\PY@tc##1{\textcolor[rgb]{1.00,0.00,0.00}{##1}}}
\expandafter\def\csname PY@tok@ge\endcsname{\let\PY@it=\textit}
\expandafter\def\csname PY@tok@gs\endcsname{\let\PY@bf=\textbf}
\expandafter\def\csname PY@tok@gp\endcsname{\let\PY@bf=\textbf\def\PY@tc##1{\textcolor[rgb]{0.00,0.00,0.50}{##1}}}
\expandafter\def\csname PY@tok@go\endcsname{\def\PY@tc##1{\textcolor[rgb]{0.53,0.53,0.53}{##1}}}
\expandafter\def\csname PY@tok@gt\endcsname{\def\PY@tc##1{\textcolor[rgb]{0.00,0.27,0.87}{##1}}}
\expandafter\def\csname PY@tok@err\endcsname{\def\PY@bc##1{\setlength{\fboxsep}{0pt}\fcolorbox[rgb]{1.00,0.00,0.00}{1,1,1}{\strut ##1}}}
\expandafter\def\csname PY@tok@kc\endcsname{\let\PY@bf=\textbf\def\PY@tc##1{\textcolor[rgb]{0.00,0.50,0.00}{##1}}}
\expandafter\def\csname PY@tok@kd\endcsname{\let\PY@bf=\textbf\def\PY@tc##1{\textcolor[rgb]{0.00,0.50,0.00}{##1}}}
\expandafter\def\csname PY@tok@kn\endcsname{\let\PY@bf=\textbf\def\PY@tc##1{\textcolor[rgb]{0.00,0.50,0.00}{##1}}}
\expandafter\def\csname PY@tok@kr\endcsname{\let\PY@bf=\textbf\def\PY@tc##1{\textcolor[rgb]{0.00,0.50,0.00}{##1}}}
\expandafter\def\csname PY@tok@bp\endcsname{\def\PY@tc##1{\textcolor[rgb]{0.00,0.50,0.00}{##1}}}
\expandafter\def\csname PY@tok@fm\endcsname{\def\PY@tc##1{\textcolor[rgb]{0.00,0.00,1.00}{##1}}}
\expandafter\def\csname PY@tok@vc\endcsname{\def\PY@tc##1{\textcolor[rgb]{0.10,0.09,0.49}{##1}}}
\expandafter\def\csname PY@tok@vg\endcsname{\def\PY@tc##1{\textcolor[rgb]{0.10,0.09,0.49}{##1}}}
\expandafter\def\csname PY@tok@vi\endcsname{\def\PY@tc##1{\textcolor[rgb]{0.10,0.09,0.49}{##1}}}
\expandafter\def\csname PY@tok@vm\endcsname{\def\PY@tc##1{\textcolor[rgb]{0.10,0.09,0.49}{##1}}}
\expandafter\def\csname PY@tok@sa\endcsname{\def\PY@tc##1{\textcolor[rgb]{0.73,0.13,0.13}{##1}}}
\expandafter\def\csname PY@tok@sb\endcsname{\def\PY@tc##1{\textcolor[rgb]{0.73,0.13,0.13}{##1}}}
\expandafter\def\csname PY@tok@sc\endcsname{\def\PY@tc##1{\textcolor[rgb]{0.73,0.13,0.13}{##1}}}
\expandafter\def\csname PY@tok@dl\endcsname{\def\PY@tc##1{\textcolor[rgb]{0.73,0.13,0.13}{##1}}}
\expandafter\def\csname PY@tok@s2\endcsname{\def\PY@tc##1{\textcolor[rgb]{0.73,0.13,0.13}{##1}}}
\expandafter\def\csname PY@tok@sh\endcsname{\def\PY@tc##1{\textcolor[rgb]{0.73,0.13,0.13}{##1}}}
\expandafter\def\csname PY@tok@s1\endcsname{\def\PY@tc##1{\textcolor[rgb]{0.73,0.13,0.13}{##1}}}
\expandafter\def\csname PY@tok@mb\endcsname{\def\PY@tc##1{\textcolor[rgb]{0.40,0.40,0.40}{##1}}}
\expandafter\def\csname PY@tok@mf\endcsname{\def\PY@tc##1{\textcolor[rgb]{0.40,0.40,0.40}{##1}}}
\expandafter\def\csname PY@tok@mh\endcsname{\def\PY@tc##1{\textcolor[rgb]{0.40,0.40,0.40}{##1}}}
\expandafter\def\csname PY@tok@mi\endcsname{\def\PY@tc##1{\textcolor[rgb]{0.40,0.40,0.40}{##1}}}
\expandafter\def\csname PY@tok@il\endcsname{\def\PY@tc##1{\textcolor[rgb]{0.40,0.40,0.40}{##1}}}
\expandafter\def\csname PY@tok@mo\endcsname{\def\PY@tc##1{\textcolor[rgb]{0.40,0.40,0.40}{##1}}}
\expandafter\def\csname PY@tok@ch\endcsname{\let\PY@it=\textit\def\PY@tc##1{\textcolor[rgb]{0.25,0.50,0.50}{##1}}}
\expandafter\def\csname PY@tok@cm\endcsname{\let\PY@it=\textit\def\PY@tc##1{\textcolor[rgb]{0.25,0.50,0.50}{##1}}}
\expandafter\def\csname PY@tok@cpf\endcsname{\let\PY@it=\textit\def\PY@tc##1{\textcolor[rgb]{0.25,0.50,0.50}{##1}}}
\expandafter\def\csname PY@tok@c1\endcsname{\let\PY@it=\textit\def\PY@tc##1{\textcolor[rgb]{0.25,0.50,0.50}{##1}}}
\expandafter\def\csname PY@tok@cs\endcsname{\let\PY@it=\textit\def\PY@tc##1{\textcolor[rgb]{0.25,0.50,0.50}{##1}}}

\def\PYZbs{\char`\\}
\def\PYZus{\char`\_}
\def\PYZob{\char`\{}
\def\PYZcb{\char`\}}
\def\PYZca{\char`\^}
\def\PYZam{\char`\&}
\def\PYZlt{\char`\<}
\def\PYZgt{\char`\>}
\def\PYZsh{\char`\#}
\def\PYZpc{\char`\%}
\def\PYZdl{\char`\$}
\def\PYZhy{\char`\-}
\def\PYZsq{\char`\'}
\def\PYZdq{\char`\"}
\def\PYZti{\char`\~}
% for compatibility with earlier versions
\def\PYZat{@}
\def\PYZlb{[}
\def\PYZrb{]}
\makeatother


    % Exact colors from NB
    \definecolor{incolor}{rgb}{0.0, 0.0, 0.5}
    \definecolor{outcolor}{rgb}{0.545, 0.0, 0.0}



    
    % Prevent overflowing lines due to hard-to-break entities
    \sloppy 
    % Setup hyperref package
    \hypersetup{
      breaklinks=true,  % so long urls are correctly broken across lines
      colorlinks=true,
      urlcolor=urlcolor,
      linkcolor=linkcolor,
      citecolor=citecolor,
      }
    % Slightly bigger margins than the latex defaults
    
    \geometry{verbose,tmargin=1in,bmargin=1in,lmargin=1in,rmargin=1in}
    
    

    \begin{document}
    
    
    \maketitle
    
    

    
    \section{Network density and
mechanics}\label{network-density-and-mechanics}

Let \([E]\) be the concentration of barbed ends and \([WA]\) the
concentration of activated nucleation promoting factors. \(k_N\),
\(k_P\), and \(k_A\) are the rates of nucleation, polymerization, and
activation respectively. The rate equations describing the changes in
barbed-end concentration over time are then:

\begin{align*}
\frac{d[E]}{dt} &= k_N[E][WA]-k_C[E] \\
\frac{d[WA]}{dt} &= -k_N[E][WA]-k_P[E][WA]+k_A\left(W_0-[WA]\right) \\
\end{align*}

\([W_0]\) is the maximum concentration of activated nucleation promoting
factors. At steady state, \(\frac{d[E]}{dt} = 0\) and
\(\frac{d[WA]}{dt} = 0\). In addition, \([E]\) is never zero, so:

\begin{align*}
0 &= k_N [E][WA] - k_C [E] \\
[WA] &= \frac{k_C}{k_N} \\
0 &= k_A (W_0 - [WA]) - k_N [E][WA] - k_P [E][WA] \\
0 &= k_A (W_0 - \frac{k_C}{k_N}) - \frac{ k_C (k_N + k_P)}{k_N} [E] \\
[E] &= \frac{k_A (W_0 k_N - k_C)}{k_C (k_N + k_P)} \\
\end{align*}

If we assume that \(k_P >> k_N\), then we obtain:

\begin{align*}
[E] &= \frac{k_A (W_0 k_N - k_C)}{k_C k_P}
\end{align*}

Requiring \((W_0 k_N - k_C) > 0\) indicates the minimum concentration of
nucleation promoting factors required to assemble a network for given
concentrations of Arp2/3 complex and capping protein, which set \(k_N\)
and \(k_C\) respectively.

\subsection{Force dependence}\label{force-dependence}

During growth, polymerization and capping rates are slowed down by the
average force \(F\) exerted on each filament at the leading edge:

\begin{align*}
k_P (F) &= k_P \exp (-F\delta / k_B T) \\
k_C (F) &= k_C \exp (-F\delta / k_B T) \\
\end{align*}

A single \(\delta\) value can be used for both expressions because an
actin monomer and a capping protein heterodimer happen to have similar
dimensions. As a result, average filament length \(\frac{k_P}{k_C}\)
does not change with network load. \(k_B\) is Boltzmann's constant, and
\(T\) the temperature. If we account for force dependence, then the
steady-state concentration of barbed ends is approximately linearly
proportional to the applied load:

\begin{align*}
E(F) &= \frac{k_A}{k_P(F)}(W_0 \frac{k_N}{k_C(F)} - 1) \\
\end{align*}

If we let \(F\delta / k_B T = f\):

\begin{align*}
E(F) &= W_0 \frac{k_A k_N}{k_P k_C} \exp (2f) - \frac{k_A}{k_P} \exp (f) \\
\end{align*}

If we assume that \(f\) is relatively small, a Taylor expansion yields:

\begin{align*}
E(F) &\approx W_0 \frac{k_A k_N}{k_P k_C} (1 + 2f) - \frac{k_A}{k_P} (1 + f) \\
E(F) &\approx W_0 \frac{k_A k_N}{k_P k_C} - \frac{k_A}{k_P} + (W_0 \frac{k_A k_N}{k_P k_C} - \frac{k_A}{k_P} + W_0 \frac{k_A k_N}{k_P k_C} )f
\end{align*}

Realizing that \(E(0) = W_0 \frac{k_A k_N}{k_P k_C} - \frac{k_A}{k_P}\),
we obtain:

\begin{align*}
E(F) &\approx E(0) + (E(0) + W_0 \frac{k_A k_N}{k_P k_C})f \\
\end{align*}

    \subsection{Stochastic model in Mueller et al., Cell
2017}\label{stochastic-model-in-mueller-et-al.-cell-2017}

The model is based on work presented in Maly and Borisy, 2001; Schaus et
al., 2007; Weichsel and Schwarz, 2010. The two-dimensional simulations
are set up as follows:

\subsubsection{Setup}\label{setup}

\(N\) filaments are initialized at uniformly random positions bounded by
\(0 <= x < L\) and \(0 <= y < \frac{dw}{2}\), where \(L\) is the length
of the leading edge and \(dw\) the width of the region where
polymerization and branching occur. The orientations of the filaments
are sampled uniformly from the \(-90^\circ <= \theta < 90^\circ\)
interval. In the paper, \(dw\) width is taken to be ten times the width
of an actin monomer (2.7 nm).

The elongation, branching, and capping rates of a given filament are
computed using heaviside functions that take its position as argument.

\subsubsection{Time evolution}\label{time-evolution}

At time \(t\), the simulation iterates over each filament that figures
out if it is going to get capped, branched, or elongate. The kinetic
rates specify Poisson distributions. For a given process, the simulation
draws a random number of its associated distribution, and the process
occurs if this number if greater than 0.5. The kinetic rates are
calculated as follows:

\paragraph{Elongation}\label{elongation}

The rate of elongation is denoted by \(\lambda_0\) and is constant over
time. This rate is zero outside the polymerization region (width
\(dw\)).

\paragraph{Branching}\label{branching}

Branching is taken to be zeroth-order with respect to the number of
barbed ends. In the model, the branching rate \(\beta\) is equal to
\(\alpha_0 a(t)\), where \(\alpha_0\) is the activation rate of a
nucleation promoting factor (NPF) present at concentration \(a(t)\). In
turn, this quantity is related to the other processes like so:

\begin{align*}
\frac{\delta D}{\delta t} &= \alpha_0 a(t) D(t) - \kappa (t) D(t) \\
\frac{\delta a}{\delta t} &= \beta_0 - \alpha_0 a(t) D(t) \\
\end{align*}

Above, \(D(t)\) is the concentration of barbed ends, \(\kappa\) the
capping rate, and \(\beta_0\) the rate of NPF recruitment to the leading
edge. The model assumes that \(\beta_0\) is much faster than branching
and capping processes, so that \(\frac{\delta a}{\delta t} \approx 0\).
Thus, the branching rate turns out to be
\(\beta = \frac{\beta_0}{D(t)}\). In the simulation, branching rates
decrease as the increasing number of barbed ends exhaust nucleation
sites.

Daughter filaments are nucleated directly off the barbed ends of other
filaments, and the angles between the daughter and mother filaments are
sampled from a normal distribution centered around \(70^\circ\) and with
a \(15^\circ\) standard deviation.

\paragraph{Capping}\label{capping}

In the simulations, filaments get capped at a rate \(\kappa_0\) when
they are in the polymerization and branching region. This rate is
significantly higher (\(\kappa_1\)) outside this region, so these
filaments quickly lose contact with the leading edge.

One oddity in this model is that the associated Poisson rate does not
take the density of barbed ends into account; capping is first-order
with respect to the number of barbed ends.

\paragraph{Force-velocity}\label{force-velocity}

The network behaves like a Brownian ratchet, so the load is shared by
the uncapped filaments:

\begin{align*}
V(F) &= V_0 \lambda_0 \delta \exp(\frac{-F \delta}{k_B T D(t)}) \\
\end{align*}

At the end of each iteration in the simulation, the leading edge moves
\(V(F) dt\), where \(dt\) is the time interval in the simulation.

\subsubsection{Implementation}\label{implementation}

One of the authors shared his Python code on
\href{https://github.com/gszep/lamellipodium}{Github}. I cloned the
repository and rewrote the simulation code to become familiar with it.
The module that does all the interesting stuff is
\href{https://jupyter.ucsf.edu:8888/edit/branched-networks/branchedNetwork.py}{muellerSixt.py}.
Below, we specific parameter values for a simulation run:

    \begin{Verbatim}[commandchars=\\\{\}]
{\color{incolor}In [{\color{incolor}1}]:} \PY{k+kn}{from} \PY{n+nn}{muellerSixt}\PY{n+nn}{.}\PY{n+nn}{py} \PY{k}{import} \PY{o}{*}
        \PY{k+kn}{from} \PY{n+nn}{numpy} \PY{k}{import} \PY{n}{pi}\PY{p}{,} \PY{n}{sin}\PY{p}{,} \PY{n}{cos}\PY{p}{,} \PY{n}{arccos}\PY{p}{,} \PY{n}{sqrt}\PY{p}{,} \PY{n}{median}\PY{p}{,} \PY{n}{convolve}\PY{p}{,} \PY{n}{diagonal}\PY{p}{,} \PY{n}{corrcoef}\PY{p}{,} \PY{n}{argmax}\PY{p}{,} \PY{n}{insert}\PY{p}{,} \PY{n}{cumsum}\PY{p}{,} \PY{n}{std}\PY{p}{,} \PY{n}{unique}\PY{p}{,} \PY{n}{savetxt}\PY{p}{,} \PY{n}{vstack}\PY{p}{,} \PY{n}{nanstd}\PY{p}{,} \PY{n}{nanmean}\PY{p}{,} \PY{n}{hstack}\PY{p}{,} \PY{n}{NaN}\PY{p}{,} \PY{n}{zeros}\PY{p}{,} \PY{n}{correlate}\PY{p}{,} \PY{n}{ones}
        \PY{k+kn}{from} \PY{n+nn}{matplotlib}\PY{n+nn}{.}\PY{n+nn}{pyplot} \PY{k}{import} \PY{o}{*}
        \PY{k+kn}{from} \PY{n+nn}{matplotlib}\PY{n+nn}{.}\PY{n+nn}{cm} \PY{k}{import} \PY{o}{*}
        \PY{o}{\PYZpc{}}\PY{k}{matplotlib} inline
        
        \PY{c+c1}{\PYZsh{} Parameter values for the simulation.}
        \PY{n}{N} \PY{o}{=} \PY{l+m+mi}{200} \PY{c+c1}{\PYZsh{} initial number of filaments}
        \PY{n}{dN} \PY{o}{=} \PY{l+m+mf}{2.7} \PY{c+c1}{\PYZsh{} length of a monomer, in nm}
        \PY{n}{filRange} \PY{o}{=} \PY{l+m+mf}{1000.0} \PY{c+c1}{\PYZsh{} width of the leading edge, in nm}
        \PY{n}{T} \PY{o}{=} \PY{l+m+mf}{20.0} \PY{c+c1}{\PYZsh{} simulation run time, in s}
        \PY{n}{dt} \PY{o}{=} \PY{l+m+mf}{0.001} \PY{c+c1}{\PYZsh{} duration of time interval, in s}
        \PY{n}{ds} \PY{o}{=} \PY{l+m+mf}{2.0} \PY{c+c1}{\PYZsh{} frame rate}
        \PY{n}{dw} \PY{o}{=} \PY{l+m+mi}{10} \PY{o}{*} \PY{n}{dN} \PY{c+c1}{\PYZsh{} width of branching zone, in nm}
        
        \PY{c+c1}{\PYZsh{} Rate functions.}
        \PY{c+c1}{\PYZsh{} Elongation}
        \PY{n}{lambdaRate} \PY{o}{=} \PY{l+m+mf}{141.0}
        \PY{k}{def} \PY{n+nf}{rLambda}\PY{p}{(}\PY{n}{x}\PY{p}{,} \PY{n}{y}\PY{p}{,} \PY{n}{t}\PY{p}{)}\PY{p}{:}
            \PY{k}{return} \PY{n}{lambdaRate} \PY{o}{*} \PY{n}{heavisidePi}\PY{p}{(}\PY{n}{y}\PY{p}{,} \PY{n}{dw}\PY{p}{)}
        
        \PY{c+c1}{\PYZsh{} Branching}
        \PY{n}{betaRate} \PY{o}{=} \PY{l+m+mf}{0.5}
        \PY{k}{def} \PY{n+nf}{rBeta}\PY{p}{(}\PY{n}{x}\PY{p}{,} \PY{n}{y}\PY{p}{,} \PY{n}{t}\PY{p}{)}\PY{p}{:}
            \PY{k}{return} \PY{n}{betaRate} \PY{o}{*} \PY{n}{heavisideTheta}\PY{p}{(}\PY{n}{y} \PY{o}{+} \PY{n}{dw} \PY{o}{/} \PY{l+m+mi}{2}\PY{p}{)}
        
        \PY{c+c1}{\PYZsh{} Capping}
        \PY{n}{kappaRate} \PY{o}{=} \PY{l+m+mf}{1.0}
        \PY{k}{def} \PY{n+nf}{rKappa}\PY{p}{(}\PY{n}{x}\PY{p}{,} \PY{n}{y}\PY{p}{,} \PY{n}{t}\PY{p}{)}\PY{p}{:}
            \PY{k}{return} \PY{l+m+mf}{1000.0} \PY{o}{*} \PY{n}{heavisideTheta}\PY{p}{(}\PY{o}{\PYZhy{}}\PY{n}{y} \PY{o}{\PYZhy{}} \PY{n}{dw} \PY{o}{/} \PY{l+m+mi}{2}\PY{p}{)} \PY{o}{+} \PY{n}{kappaRate} \PY{o}{*} \PY{n}{heavisidePi}\PY{p}{(}\PY{n}{y}\PY{p}{,} \PY{n}{dw}\PY{p}{)}
        
        \PY{c+c1}{\PYZsh{} Plot rate functions.}
        \PY{n}{plot}\PY{p}{(}\PY{n}{arange}\PY{p}{(}\PY{o}{\PYZhy{}}\PY{l+m+mi}{100}\PY{p}{,} \PY{l+m+mi}{100}\PY{p}{,} \PY{l+m+mf}{0.01}\PY{p}{)}\PY{p}{,} \PY{p}{[}\PY{n}{rLambda}\PY{p}{(}\PY{l+m+mi}{0}\PY{p}{,} \PY{n}{y}\PY{p}{,} \PY{l+m+mi}{0}\PY{p}{)} \PY{o}{/} \PY{l+m+mi}{100} \PY{k}{for} \PY{n}{y} \PY{o+ow}{in} \PY{n}{arange}\PY{p}{(}\PY{o}{\PYZhy{}}\PY{l+m+mi}{100}\PY{p}{,} \PY{l+m+mi}{100}\PY{p}{,} \PY{l+m+mf}{0.01}\PY{p}{)}\PY{p}{]}\PY{p}{,} \PY{l+s+s1}{\PYZsq{}}\PY{l+s+s1}{g}\PY{l+s+s1}{\PYZsq{}}\PY{p}{)}
        \PY{n}{plot}\PY{p}{(}\PY{n}{arange}\PY{p}{(}\PY{o}{\PYZhy{}}\PY{l+m+mi}{100}\PY{p}{,} \PY{l+m+mi}{100}\PY{p}{,} \PY{l+m+mf}{0.01}\PY{p}{)}\PY{p}{,} \PY{p}{[}\PY{n}{rBeta}\PY{p}{(}\PY{l+m+mi}{0}\PY{p}{,} \PY{n}{y}\PY{p}{,} \PY{l+m+mi}{0}\PY{p}{)} \PY{k}{for} \PY{n}{y} \PY{o+ow}{in} \PY{n}{arange}\PY{p}{(}\PY{o}{\PYZhy{}}\PY{l+m+mi}{100}\PY{p}{,} \PY{l+m+mi}{100}\PY{p}{,} \PY{l+m+mf}{0.01}\PY{p}{)}\PY{p}{]}\PY{p}{,} \PY{l+s+s1}{\PYZsq{}}\PY{l+s+s1}{b}\PY{l+s+s1}{\PYZsq{}}\PY{p}{)}
        \PY{n}{plot}\PY{p}{(}\PY{n}{arange}\PY{p}{(}\PY{o}{\PYZhy{}}\PY{l+m+mi}{100}\PY{p}{,} \PY{l+m+mi}{100}\PY{p}{,} \PY{l+m+mf}{0.01}\PY{p}{)}\PY{p}{,} \PY{p}{[}\PY{n}{rKappa}\PY{p}{(}\PY{l+m+mi}{0}\PY{p}{,} \PY{n}{y}\PY{p}{,} \PY{l+m+mi}{0}\PY{p}{)} \PY{o}{/} \PY{l+m+mi}{1000} \PY{k}{for} \PY{n}{y} \PY{o+ow}{in} \PY{n}{arange}\PY{p}{(}\PY{o}{\PYZhy{}}\PY{l+m+mi}{100}\PY{p}{,} \PY{l+m+mi}{100}\PY{p}{,} \PY{l+m+mf}{0.01}\PY{p}{)}\PY{p}{]}\PY{p}{,} \PY{l+s+s1}{\PYZsq{}}\PY{l+s+s1}{r}\PY{l+s+s1}{\PYZsq{}}\PY{p}{)}
\end{Verbatim}


\begin{Verbatim}[commandchars=\\\{\}]
{\color{outcolor}Out[{\color{outcolor}1}]:} [<matplotlib.lines.Line2D at 0x7f6d42145240>]
\end{Verbatim}
            
    \begin{center}
    \adjustimage{max size={0.9\linewidth}{0.9\paperheight}}{output_2_1.png}
    \end{center}
    { \hspace*{\fill} \\}
    
    We then initialize arrays indicating where the filaments are and their
orientation. Then we create a branched network object and run the
simulation.

    \begin{Verbatim}[commandchars=\\\{\}]
{\color{incolor}In [{\color{incolor}2}]:} \PY{c+c1}{\PYZsh{} Sample from uniform distribution given N filaments.}
        \PY{n}{thetaDist} \PY{o}{=} \PY{n}{uniform}\PY{p}{(}\PY{o}{\PYZhy{}}\PY{n}{pi}\PY{p}{,} \PY{n}{pi}\PY{p}{,} \PY{n}{size} \PY{o}{=} \PY{n}{N}\PY{p}{)}
        \PY{n}{xDist} \PY{o}{=} \PY{n}{uniform}\PY{p}{(}\PY{n}{low} \PY{o}{=} \PY{l+m+mf}{0.0}\PY{p}{,} \PY{n}{high} \PY{o}{=} \PY{n}{filRange}\PY{p}{,} \PY{n}{size} \PY{o}{=} \PY{n}{N}\PY{p}{)}
        \PY{n}{yDist} \PY{o}{=} \PY{n}{uniform}\PY{p}{(}\PY{n}{low} \PY{o}{=} \PY{l+m+mf}{0.0}\PY{p}{,} \PY{n}{high} \PY{o}{=} \PY{n}{dw} \PY{o}{/} \PY{l+m+mf}{2.0}\PY{p}{,} \PY{n}{size} \PY{o}{=} \PY{n}{N}\PY{p}{)}
        
        \PY{n}{xInit} \PY{o}{=} \PY{n}{array}\PY{p}{(}\PY{p}{[}\PY{p}{[}\PY{n}{x}\PY{p}{,} \PY{n}{y}\PY{p}{]} \PY{k}{for} \PY{n}{x}\PY{p}{,} \PY{n}{y} \PY{o+ow}{in} \PY{n+nb}{zip}\PY{p}{(}\PY{n}{xDist}\PY{p}{,} \PY{n}{yDist}\PY{p}{)}\PY{p}{]}\PY{p}{)}
        \PY{n}{dxInit} \PY{o}{=} \PY{n}{array}\PY{p}{(}\PY{p}{[}\PY{p}{[}\PY{n}{dN} \PY{o}{*} \PY{n}{sin}\PY{p}{(}\PY{n}{theta}\PY{p}{)}\PY{p}{,} \PY{n}{dN} \PY{o}{*} \PY{n}{cos}\PY{p}{(}\PY{n}{theta}\PY{p}{)}\PY{p}{]} \PY{k}{for} \PY{n}{theta} \PY{o+ow}{in} \PY{n}{thetaDist}\PY{p}{]}\PY{p}{)}
        
        \PY{c+c1}{\PYZsh{} Initialize object.}
        \PY{n}{n} \PY{o}{=} \PY{n}{network}\PY{p}{(}\PY{n}{rLambda}\PY{p}{,} \PY{n}{rBeta}\PY{p}{,} \PY{n}{rKappa}\PY{p}{,} \PY{n}{xSeed} \PY{o}{=} \PY{n}{xInit}\PY{p}{,} \PY{n}{dxSeed} \PY{o}{=} \PY{n}{dxInit}\PY{p}{,} \PY{n}{branchSigma} \PY{o}{=} \PY{l+m+mf}{15.0} \PY{o}{*} \PY{n}{pi} \PY{o}{/} \PY{l+m+mi}{180}\PY{p}{,} \PY{n}{forceDirection}\PY{o}{=} \PY{k+kc}{True}\PY{p}{,} \PY{n}{recordHistory} \PY{o}{=} \PY{k+kc}{True}\PY{p}{)}
        \PY{n}{n}\PY{o}{.}\PY{n}{exportData}\PY{p}{(}\PY{n}{dt}\PY{p}{,} \PY{n}{ds}\PY{p}{,} \PY{n}{n}\PY{o}{.}\PY{n}{tElapsed} \PY{o}{+} \PY{n}{T}\PY{p}{,} \PY{n}{Fext} \PY{o}{=} \PY{l+m+mf}{1.0e\PYZhy{}1}\PY{p}{)}
        
        \PY{c+c1}{\PYZsh{} Plot network.}
        \PY{n}{xFil} \PY{o}{=} \PY{n}{n}\PY{o}{.}\PY{n}{getPositions}\PY{p}{(}\PY{n}{n}\PY{o}{.}\PY{n}{Monomers}\PY{p}{)}
        \PY{n}{xBranch} \PY{o}{=} \PY{n}{n}\PY{o}{.}\PY{n}{getPositions}\PY{p}{(}\PY{n}{n}\PY{o}{.}\PY{n}{Branches}\PY{p}{)}
        \PY{n}{xCap} \PY{o}{=} \PY{n}{n}\PY{o}{.}\PY{n}{getPositions}\PY{p}{(}\PY{n}{n}\PY{o}{.}\PY{n}{Caps}\PY{p}{)}
        \PY{n}{figure}\PY{p}{(}\PY{n}{figsize}\PY{o}{=}\PY{p}{(}\PY{l+m+mi}{20}\PY{p}{,}\PY{l+m+mi}{5}\PY{p}{)}\PY{p}{)}
        \PY{n}{plot}\PY{p}{(}\PY{n}{xFil}\PY{o}{.}\PY{n}{T}\PY{p}{[}\PY{l+m+mi}{1}\PY{p}{]}\PY{p}{,} \PY{n}{xFil}\PY{o}{.}\PY{n}{T}\PY{p}{[}\PY{l+m+mi}{0}\PY{p}{]}\PY{p}{,} \PY{l+s+s1}{\PYZsq{}}\PY{l+s+s1}{g}\PY{l+s+s1}{\PYZsq{}}\PY{p}{,} \PY{n}{marker} \PY{o}{=} \PY{l+s+s2}{\PYZdq{}}\PY{l+s+s2}{.}\PY{l+s+s2}{\PYZdq{}}\PY{p}{,} \PY{n}{linewidth} \PY{o}{=} \PY{l+m+mi}{0}\PY{p}{,} \PY{n}{ms} \PY{o}{=} \PY{l+m+mi}{5}\PY{p}{,} \PY{n}{alpha} \PY{o}{=} \PY{l+m+mf}{0.2}\PY{p}{)}
        \PY{k}{if} \PY{n+nb}{len}\PY{p}{(}\PY{n}{xBranch}\PY{p}{)} \PY{o}{!=} \PY{l+m+mi}{0}\PY{p}{:}
            \PY{n}{plot}\PY{p}{(}\PY{n}{xBranch}\PY{o}{.}\PY{n}{T}\PY{p}{[}\PY{l+m+mi}{1}\PY{p}{]}\PY{p}{,} \PY{n}{xBranch}\PY{o}{.}\PY{n}{T}\PY{p}{[}\PY{l+m+mi}{0}\PY{p}{]}\PY{p}{,} \PY{l+s+s1}{\PYZsq{}}\PY{l+s+s1}{\PYZsh{}2737ff}\PY{l+s+s1}{\PYZsq{}}\PY{p}{,} \PY{n}{marker} \PY{o}{=} \PY{l+s+s2}{\PYZdq{}}\PY{l+s+s2}{.}\PY{l+s+s2}{\PYZdq{}}\PY{p}{,} \PY{n}{linewidth} \PY{o}{=} \PY{l+m+mi}{0}\PY{p}{,} \PY{n}{ms} \PY{o}{=} \PY{l+m+mi}{6}\PY{p}{,} \PY{n}{alpha} \PY{o}{=} \PY{l+m+mf}{0.5}\PY{p}{)}
        \PY{k}{if} \PY{n+nb}{len}\PY{p}{(}\PY{n}{xCap}\PY{p}{)} \PY{o}{!=} \PY{l+m+mi}{0}\PY{p}{:}
            \PY{n}{plot}\PY{p}{(}\PY{n}{xCap}\PY{o}{.}\PY{n}{T}\PY{p}{[}\PY{l+m+mi}{1}\PY{p}{]}\PY{p}{,} \PY{n}{xCap}\PY{o}{.}\PY{n}{T}\PY{p}{[}\PY{l+m+mi}{0}\PY{p}{]}\PY{p}{,} \PY{l+s+s1}{\PYZsq{}}\PY{l+s+s1}{\PYZsh{}ff0000}\PY{l+s+s1}{\PYZsq{}}\PY{p}{,} \PY{n}{marker} \PY{o}{=} \PY{l+s+s2}{\PYZdq{}}\PY{l+s+s2}{.}\PY{l+s+s2}{\PYZdq{}}\PY{p}{,} \PY{n}{linewidth} \PY{o}{=} \PY{l+m+mi}{0}\PY{p}{,} \PY{n}{ms} \PY{o}{=} \PY{l+m+mi}{6}\PY{p}{,} \PY{n}{alpha} \PY{o}{=} \PY{l+m+mf}{0.5}\PY{p}{)}
        \PY{n}{xlabel}\PY{p}{(}\PY{l+s+sa}{r}\PY{l+s+s2}{\PYZdq{}}\PY{l+s+s2}{Distance, \PYZdl{}x\PYZdl{} / nm}\PY{l+s+s2}{\PYZdq{}}\PY{p}{,} \PY{n}{fontsize} \PY{o}{=} \PY{l+m+mi}{18}\PY{p}{)}
        \PY{n}{ylabel}\PY{p}{(}\PY{l+s+sa}{r}\PY{l+s+s2}{\PYZdq{}}\PY{l+s+s2}{Distance, \PYZdl{}y\PYZdl{} / nm}\PY{l+s+s2}{\PYZdq{}}\PY{p}{,} \PY{n}{fontsize} \PY{o}{=} \PY{l+m+mi}{18}\PY{p}{)}
        \PY{n}{ylim}\PY{p}{(}\PY{l+m+mi}{0}\PY{p}{,} \PY{n}{n}\PY{o}{.}\PY{n}{xBoundary}\PY{p}{)}\PY{p}{;} \PY{n}{xlim}\PY{p}{(}\PY{l+m+mi}{0}\PY{p}{,} \PY{l+m+mi}{3000}\PY{p}{)}
\end{Verbatim}


    \begin{Verbatim}[commandchars=\\\{\}]
/home/jiongyi/anaconda3/lib/python3.6/site-packages/matplotlib/cbook/deprecation.py:106: MatplotlibDeprecationWarning: Adding an axes using the same arguments as a previous axes currently reuses the earlier instance.  In a future version, a new instance will always be created and returned.  Meanwhile, this warning can be suppressed, and the future behavior ensured, by passing a unique label to each axes instance.
  warnings.warn(message, mplDeprecation, stacklevel=1)

    \end{Verbatim}

\begin{Verbatim}[commandchars=\\\{\}]
{\color{outcolor}Out[{\color{outcolor}2}]:} (0, 3000)
\end{Verbatim}
            
    \begin{center}
    \adjustimage{max size={0.9\linewidth}{0.9\paperheight}}{output_4_2.png}
    \end{center}
    { \hspace*{\fill} \\}
    

    % Add a bibliography block to the postdoc
    
    
    
    \end{document}
